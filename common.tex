%!TEX TS-program = arara
%
% File   : common.tex
% Author : ɛntiˈtɛːt.kaɪ̯
% Date   : 2016-03-20
%
%---+----1----+----2----+----3----+----4----+----5----+----6----+----7----+----8
% arara config
%   arara: lualatex
%---+----1----+----2----+----3----+----4----+----5----+----6----+----7----+----8



\documentclass[a4paper,12pt]{article}
\documentclass[a4paper,12pt]{report}



	\input{packages}     % common packages
	\input{definitions}  % common definitions


	% Titelei
		\title{\fontspec{Times New Roman}common}
		\author{\fontspec{Times New Roman}ɛntiˈtɛːt.kaɪ̯}
		\date{%
			{\fontspec{Times New Roman}19. März 2016} \\
			{\bf---} \\[0.2\baselineskip]
			\fontspec{Times New Roman}\today}



\begin{document}



	% List of todos
		\listoftodos

	% Schmutztitel (Vorderseite)
		\includepdf{a4schmutztitel}

	% Titelseite
		\maketitle

	% Zusammenfassung
		%%
% File   : zusammenfassung.tex
% Author : ɛntiˈtɛːt.kaɪ̯
% Date   : 2016-03-20
%
%---+----1----+----2----+----3----+----4----+----5----+----6----+----7----+----8



\begin{abstract}
	\todo[inline,caption={Zusammenfassung}]{Zweck des Projekts.}
\end{abstract}



%---+----1----+----2----+----3----+----4----+----5----+----6----+----7----+----8
% vim:wrap:noet:ts=2 sw=2 sts=2

		%
% File   : zusammenfassung.tex
% Author : ɛntiˈtɛːt.kaɪ̯
% Date   : 2016-03-20
%
%---+----1----+----2----+----3----+----4----+----5----+----6----+----7----+----8



\begin{abstract}
	\todo[inline,caption={Zusammenfassung}]{Zweck des Projekts.}
\end{abstract}



%---+----1----+----2----+----3----+----4----+----5----+----6----+----7----+----8
% vim:wrap:noet:ts=2 sw=2 sts=2


	% Verzeichnisse
		\tableofcontents  % Inhaltsverzeichnis

	% Inhalt
		\chapter{Installation}\input{installation}
		\chapter{Pakete}%
% File   : definitionen.tex
% Author : ɛntiˈtɛːt.kaɪ̯
% Date   : 2016-03-21
%
%---+----1----+----2----+----3----+----4----+----5----+----6----+----7----+----8% LaTeX Metadaten
	\label{sec:Pakete}
	\index{Paket}
%---+----1----+----2----+----3----+----4----+----5----+----6----+----7----+----8



\section{Paket polyglossia von François Charette (Maintainer: Arthur Reutenauer)}\input{packages/polyglossia}
\section{Paket placeins von Donald Arseneau}\input{packages/placeins}
\section{Paket pdfpages von Andreas Matthias}\input{packages/pdfpages}
\section{Paket caption von Axel Sommerfeldt}\input{packages/caption}
\section{Paket tabularx von David Carlisle}\input{packages/tabularx}
\section{Paket listingsutf8 von Heiko Oberdiek}\input{packages/listingsutf8}
\section{Paket todonotes von Henrik Skov Midtiby}\input{packages/todonotes}
\section{Paket gitinfo2 von Brent Longborough}%
% File   : gitinfo2.tex
% Author : ɛntiˈtɛːt.kaɪ̯
% Date   : 2016-03-20
%
%---+----1----+----2----+----3----+----4----+----5----+----6----+----7----+----8% LaTeX Metadaten
	\label{sec:Gitinfo2}
	\index{gitinfo2}
	\index{Paket!gitinfo2}
%---+----1----+----2----+----3----+----4----+----5----+----6----+----7----+----8



\begin{table}[b]
	\caption{Git Metadaten}
	\footnotesize
	\begin{tabularx}{\textwidth}{lFF} \hline
		Makro & Inhalt & Bemerkung \\ \hline
		\verb_\gitAbbrevHash_ &
			\gitAbbrevHash &
			Hash (7 Zeichen) \\
		\verb_\gitHash_ &
			\gitAbbrevHash\ldots &
			Hash (40 Zeichen, hier in verkürzter Darstellung) \\ \hline
		\verb_\gitAuthorName_ &
			\fontspec{Times New Roman}\gitAuthorName &
			Autor \\
		\verb_\gitAuthorEmail_ &
			\gitAuthorEmail &
			E-Mail (Autor) \\
		\verb_\gitAuthorDate_ &
			\gitAuthorDate &
			Datum (commit) \\
		\verb_\gitAuthorIsoDate_ &
			\gitAuthorIsoDate &
			ISO-Datum (commit) \\
		\verb_\gitAuthorUnixDate_ &
			\gitAuthorUnixDate &
			Unix-Datum (commit) \\ \hline
		\verb_\gitCommitterName_ &
			\fontspec{Times New Roman}\gitCommitterName &
			Committer \\
		\verb_\gitCommitterEmail_ &
			\gitCommitterEmail &
			E-Mail (Committer) \\
		\verb_\gitCommitterDate_ &
			\gitCommitterDate &
			Committer-Datum \\
		\verb_\gitCommitterIsoDate_ &
			\gitCommitterIsoDate &
			Committer-ISO-Datum \\
		\verb_\gitCommitterUnixDate_ &
			\gitCommitterUnixDate &
			Committer-Unix-Datum \\ \hline
		\verb_\gitReferences_ &
			\gitReferences &
			Referenzen (Tags, Branches) \\
		\verb_\gitBranch_ &
			\gitBranch &
			Aktueller Branch \\ \hline
		\verb_\gitVtag_ &
			\gitVtag &
			\\
		\verb_\gitVtags_ &
			\gitVtags &
			\\
		\verb_\gitVtagn_ &
			\gitVtagn &
			\\ \hline
		\verb_\gitFirstTagDescribe_ &
			\gitFirstTagDescribe &
			\\
		\verb_\gitDirty_ &
			\gitDirty &
			\\
		\verb_\gitRel_ &
			\gitRel &
			Release-Nummer \\
		\verb_\gitRels_ &
			\gitRels &
			\\
		\verb_\gitReln_ &
			\gitReln &
			\\
		\verb_\gitRoff_ &
			\gitRoff &
			Anzahl der commits zwischen HEAD und dem letzten Release \\
		\verb_\gitTags_ &
			\gitTags &
			Liste aller Tags \\
		\verb_\gitDescribe_ &
			\gitDescribe &
			\\ \hline
	\end{tabularx}
\end{table}
\FloatBarrier



%---+----1----+----2----+----3----+----4----+----5----+----6----+----7----+----8
% vim:wrap:noet:ts=2 sw=2 sts=2

\section{Paket xspace von David Carlisle und Morten Høgholm}\input{packages/xspace}
\section{Paket hologo von Heiko Oberdiek}\input{packages/hologo}
\section{Paket imakeidx von Enrico Gregorio}\input{packages/imakeidx}
\section{Paket hyperref von Sebastian Rahtz und Heiko Oberdiek}\input{packages/hyperref}



%---+----1----+----2----+----3----+----4----+----5----+----6----+----7----+----8
% vim:nowrap:noet:ts=2 sw=2 sts=2

			\section{Paket polyglossia von François Charette (Maintainer: Arthur
				Reutenauer)}
				\input{packages/polyglossia}
			\section{Paket placeins von Donald Arseneau}
				\input{packages/placeins}
			\section{Paket pdfpages von Andreas Matthias}
				\input{packages/pdfpages}
			\section{Paket caption von Axel Sommerfeldt}
				\input{packages/caption}
			\section{Paket tabularx von David Carlisle}
				\input{packages/tabularx}
			\section{Paket listingsutf8 von Heiko Oberdiek}
				\input{packages/listingsutf8}
			\section{Paket todonotes von Henrik Skov Midtiby}
				\input{packages/todonotes}
			\section{Paket gitinfo2 von Brent Longborough}
				%
% File   : gitinfo2.tex
% Author : ɛntiˈtɛːt.kaɪ̯
% Date   : 2016-03-20
%
%---+----1----+----2----+----3----+----4----+----5----+----6----+----7----+----8% LaTeX Metadaten
	\label{sec:Gitinfo2}
	\index{gitinfo2}
	\index{Paket!gitinfo2}
%---+----1----+----2----+----3----+----4----+----5----+----6----+----7----+----8



\begin{table}[b]
	\caption{Git Metadaten}
	\footnotesize
	\begin{tabularx}{\textwidth}{lFF} \hline
		Makro & Inhalt & Bemerkung \\ \hline
		\verb_\gitAbbrevHash_ &
			\gitAbbrevHash &
			Hash (7 Zeichen) \\
		\verb_\gitHash_ &
			\gitAbbrevHash\ldots &
			Hash (40 Zeichen, hier in verkürzter Darstellung) \\ \hline
		\verb_\gitAuthorName_ &
			\fontspec{Times New Roman}\gitAuthorName &
			Autor \\
		\verb_\gitAuthorEmail_ &
			\gitAuthorEmail &
			E-Mail (Autor) \\
		\verb_\gitAuthorDate_ &
			\gitAuthorDate &
			Datum (commit) \\
		\verb_\gitAuthorIsoDate_ &
			\gitAuthorIsoDate &
			ISO-Datum (commit) \\
		\verb_\gitAuthorUnixDate_ &
			\gitAuthorUnixDate &
			Unix-Datum (commit) \\ \hline
		\verb_\gitCommitterName_ &
			\fontspec{Times New Roman}\gitCommitterName &
			Committer \\
		\verb_\gitCommitterEmail_ &
			\gitCommitterEmail &
			E-Mail (Committer) \\
		\verb_\gitCommitterDate_ &
			\gitCommitterDate &
			Committer-Datum \\
		\verb_\gitCommitterIsoDate_ &
			\gitCommitterIsoDate &
			Committer-ISO-Datum \\
		\verb_\gitCommitterUnixDate_ &
			\gitCommitterUnixDate &
			Committer-Unix-Datum \\ \hline
		\verb_\gitReferences_ &
			\gitReferences &
			Referenzen (Tags, Branches) \\
		\verb_\gitBranch_ &
			\gitBranch &
			Aktueller Branch \\ \hline
		\verb_\gitVtag_ &
			\gitVtag &
			\\
		\verb_\gitVtags_ &
			\gitVtags &
			\\
		\verb_\gitVtagn_ &
			\gitVtagn &
			\\ \hline
		\verb_\gitFirstTagDescribe_ &
			\gitFirstTagDescribe &
			\\
		\verb_\gitDirty_ &
			\gitDirty &
			\\
		\verb_\gitRel_ &
			\gitRel &
			Release-Nummer \\
		\verb_\gitRels_ &
			\gitRels &
			\\
		\verb_\gitReln_ &
			\gitReln &
			\\
		\verb_\gitRoff_ &
			\gitRoff &
			Anzahl der commits zwischen HEAD und dem letzten Release \\
		\verb_\gitTags_ &
			\gitTags &
			Liste aller Tags \\
		\verb_\gitDescribe_ &
			\gitDescribe &
			\\ \hline
	\end{tabularx}
\end{table}
\FloatBarrier



%---+----1----+----2----+----3----+----4----+----5----+----6----+----7----+----8
% vim:wrap:noet:ts=2 sw=2 sts=2

			\section{Paket hyperref von Sebastian Rahtz und Heiko Oberdiek}
				\input{packages/hyperref}
		\chapter{Definitionen}%
% File   : definitionen.tex
% Author : ɛntiˈtɛːt.kaɪ̯
% Date   : 2016-03-21
%
%---+----1----+----2----+----3----+----4----+----5----+----6----+----7----+----8% LaTeX Metadaten
	\label{sec:Definitionen}
	\index{Definition}
%---+----1----+----2----+----3----+----4----+----5----+----6----+----7----+----8



\section{Konfiguration des polyglossia Pakets}\input{definitions/polyglossia}
\section{Definitionen diverser Farben}
	\subsection{Ethan Schoonovers Farbschema „solarized“}%
% File   : solarized.tex
% Author : ɛntiˈtɛːt.kaɪ̯
% Date   : 2016-03-20
%
%---+----1----+----2----+----3----+----4----+----5----+----6----+----7----+----8% LaTeX Metadaten
	\label{sec:Solarized}
	\index{solarized}
	\index{Farbschema!solarized}
%---+----1----+----2----+----3----+----4----+----5----+----6----+----7----+----8



Die folgenden 16~Farbfelder zeigen die Farben von Ethan Schoonovers \href{http://ethanschoonover.com/solarized}{Farbschema solarized}.

\begin{figure}[h]
	\centering
	\begin{tikzpicture}
		\draw (1.45,1.6) -- (1.45,1.9) node [above] {\scriptsize Background Tones};
		\draw (0.3,1.6) -- (2.6,1.6);
		\draw (6.25,1.6) -- (6.25,1.9) node [above] {\scriptsize Content Tones};
		\draw (3.5,1.6) -- (9,1.6);
		\draw (11.05,1.6) -- (11.05,1.9) node [above] {\scriptsize Background Tones};
		\draw (9.9,1.6) -- (12.2,1.6);
		\draw [fill=Solarized-Base03] (0,0) rectangle (1.3,1.3);
		\draw [fill=Solarized-Base02] (1.6,0) rectangle (2.9,1.3);
		\draw [fill=Solarized-Base01] (3.2,0) rectangle (4.5,1.3);
		\draw [fill=Solarized-Base00] (4.8,0) rectangle (6.1,1.3);
		\draw [fill=Solarized-Base0] (6.4,0) rectangle (7.7,1.3);
		\draw [fill=Solarized-Base1] (8,0) rectangle (9.3,1.3);
		\draw [fill=Solarized-Base2] (9.6,0) rectangle (10.9,1.3);
		\draw [fill=Solarized-Base3] (11.2,0) rectangle (12.5,1.3);
		\draw (0.65,0) node [anchor=north] {\scriptsize Base03};
		\draw (2.25,0) node [anchor=north] {\scriptsize Base02};
		\draw (3.85,0) node [anchor=north] {\scriptsize Base01};
		\draw (5.45,0) node [anchor=north] {\scriptsize Base00};
		\draw (7.05,0) node [anchor=north] {\scriptsize Base0};
		\draw (8.65,0) node [anchor=north] {\scriptsize Base1};
		\draw (10.25,0) node [anchor=north] {\scriptsize Base2};
		\draw (11.85,0) node [anchor=north] {\scriptsize Base3};
	\end{tikzpicture}

	~\\
	\begin{tikzpicture}
		\draw (6.25,1.6) -- (6.25,1.9) node [above] {\scriptsize Accent Colors};
		\draw (0.3,1.6) -- (12.2,1.6);
		\draw [fill=Solarized-Yellow] (0,0) rectangle (1.3,1.3);
		\draw [fill=Solarized-Orange] (1.6,0) rectangle (2.9,1.3);
		\draw [fill=Solarized-Red] (3.2,0) rectangle (4.5,1.3);
		\draw [fill=Solarized-Magenta] (4.8,0) rectangle (6.1,1.3);
		\draw [fill=Solarized-Violet] (6.4,0) rectangle (7.7,1.3);
		\draw [fill=Solarized-Blue] (8,0) rectangle (9.3,1.3);
		\draw [fill=Solarized-Cyan] (9.6,0) rectangle (10.9,1.3);
		\draw [fill=Solarized-Green] (11.2,0) rectangle (12.5,1.3);
		\draw (0.65,0) node [anchor=north] {\scriptsize Yellow};
		\draw (2.25,0) node [anchor=north] {\scriptsize Orange};
		\draw (3.85,0) node [anchor=north] {\scriptsize Red};
		\draw (5.45,0) node [anchor=north] {\scriptsize Magenta};
		\draw (7.05,0) node [anchor=north] {\scriptsize Violet};
		\draw (8.65,0) node [anchor=north] {\scriptsize Blue};
		\draw (10.25,0) node [anchor=north] {\scriptsize Cyan};
		\draw (11.85,0) node [anchor=north] {\scriptsize Green};
	\end{tikzpicture}
	\caption{Farbschema solarized}
\end{figure}

Die Farben können verwendet werden, indem den Farbnamen unter den obigen Farbfeldern die Zeichenkette \texttt{Solarized-} vorangestellt wird. \\

\paragraph*{Beispiel:} \verb_{\color{Solarized-Blue}blauen}_ erzeugt einen {\color{Solarized-Blue}blauen} Text.



%---+----1----+----2----+----3----+----4----+----5----+----6----+----7----+----8
% vim:wrap:noet:ts=2 sw=2 sts=2

	\subsection{Apple (Finder) Tag-Farben}%
% File   : findertags.tex
% Author : ɛntiˈtɛːt.kaɪ̯
% Date   : 2016-03-20
%
%---+----1----+----2----+----3----+----4----+----5----+----6----+----7----+----8% LaTeX Metadaten
	\label{sec:FinderTags}
	\index{Tags}
	\index{Farbschema!Tags}
%---+----1----+----2----+----3----+----4----+----5----+----6----+----7----+----8



\begin{figure}[h]
	\centering
	\begin{tikzpicture}
		%\draw (5.45,1.6) -- (5.45,1.9) node [above] {\scriptsize Standard-Farben};
		%\draw (0.3,1.6) -- (10.6,1.6);
		\draw [fill=Finder-Tag-Red] (0,0) rectangle (1.3,1.3);
		\draw [fill=Finder-Tag-Orange] (1.6,0) rectangle (2.9,1.3);
		\draw [fill=Finder-Tag-Yellow] (3.2,0) rectangle (4.5,1.3);
		\draw [fill=Finder-Tag-Green] (4.8,0) rectangle (6.1,1.3);
		\draw [fill=Finder-Tag-Blue] (6.4,0) rectangle (7.7,1.3);
		\draw [fill=Finder-Tag-Violet] (8,0) rectangle (9.3,1.3);
		\draw [fill=Finder-Tag-Gray] (9.6,0) rectangle (10.9,1.3);
		\draw (0.65,0) node [anchor=north] {\scriptsize Red};
		\draw (2.25,0) node [anchor=north] {\scriptsize Orange};
		\draw (3.85,0) node [anchor=north] {\scriptsize Yellow};
		\draw (5.45,0) node [anchor=north] {\scriptsize Green};
		\draw (7.05,0) node [anchor=north] {\scriptsize Blue};
		\draw (8.65,0) node [anchor=north] {\scriptsize Violet};
		\draw (10.25,0) node [anchor=north] {\scriptsize Gray};
	\end{tikzpicture}
	\caption{Farben der Tags in Apples Finder}
\end{figure}
\FloatBarrier

Die Farben können verwendet werden, indem den Farbnamen unter den obigen Farbfeldern die Zeichenkette \texttt{Finder-Tag-} vorangestellt wird. \\

\paragraph*{Beispiel:} \verb_{\color{Finder-Tag-Blue}blauen}_ erzeugt einen {\color{Finder-Tag-Blue}blauen} Text.



%---+----1----+----2----+----3----+----4----+----5----+----6----+----7----+----8
% vim:wrap:noet:ts=2 sw=2 sts=2

\section{Konfiguration des tabularx Pakets}\input{definitions/tabularx}
\section{Konfiguration des listingsutf8 Pakets}\input{definitions/listingsutf8}
\section{Konfiguration des todonotes Pakets}\input{definitions/todonotes}
\section{Definitionen diverser Makros}\input{definitions/makros}



%---+----1----+----2----+----3----+----4----+----5----+----6----+----7----+----8
% vim:nowrap:noet:ts=2 sw=2 sts=2

			\section{Konfiguration des polyglossia Pakets}
				\input{definitions/polyglossia}
			\section{Definitionen diverser Farben}
				\subsection{Ethan Schoonovers Farbschema „solarized“}
					%
% File   : solarized.tex
% Author : ɛntiˈtɛːt.kaɪ̯
% Date   : 2016-03-20
%
%---+----1----+----2----+----3----+----4----+----5----+----6----+----7----+----8% LaTeX Metadaten
	\label{sec:Solarized}
	\index{solarized}
	\index{Farbschema!solarized}
%---+----1----+----2----+----3----+----4----+----5----+----6----+----7----+----8



Die folgenden 16~Farbfelder zeigen die Farben von Ethan Schoonovers \href{http://ethanschoonover.com/solarized}{Farbschema solarized}.

\begin{figure}[h]
	\centering
	\begin{tikzpicture}
		\draw (1.45,1.6) -- (1.45,1.9) node [above] {\scriptsize Background Tones};
		\draw (0.3,1.6) -- (2.6,1.6);
		\draw (6.25,1.6) -- (6.25,1.9) node [above] {\scriptsize Content Tones};
		\draw (3.5,1.6) -- (9,1.6);
		\draw (11.05,1.6) -- (11.05,1.9) node [above] {\scriptsize Background Tones};
		\draw (9.9,1.6) -- (12.2,1.6);
		\draw [fill=Solarized-Base03] (0,0) rectangle (1.3,1.3);
		\draw [fill=Solarized-Base02] (1.6,0) rectangle (2.9,1.3);
		\draw [fill=Solarized-Base01] (3.2,0) rectangle (4.5,1.3);
		\draw [fill=Solarized-Base00] (4.8,0) rectangle (6.1,1.3);
		\draw [fill=Solarized-Base0] (6.4,0) rectangle (7.7,1.3);
		\draw [fill=Solarized-Base1] (8,0) rectangle (9.3,1.3);
		\draw [fill=Solarized-Base2] (9.6,0) rectangle (10.9,1.3);
		\draw [fill=Solarized-Base3] (11.2,0) rectangle (12.5,1.3);
		\draw (0.65,0) node [anchor=north] {\scriptsize Base03};
		\draw (2.25,0) node [anchor=north] {\scriptsize Base02};
		\draw (3.85,0) node [anchor=north] {\scriptsize Base01};
		\draw (5.45,0) node [anchor=north] {\scriptsize Base00};
		\draw (7.05,0) node [anchor=north] {\scriptsize Base0};
		\draw (8.65,0) node [anchor=north] {\scriptsize Base1};
		\draw (10.25,0) node [anchor=north] {\scriptsize Base2};
		\draw (11.85,0) node [anchor=north] {\scriptsize Base3};
	\end{tikzpicture}

	~\\
	\begin{tikzpicture}
		\draw (6.25,1.6) -- (6.25,1.9) node [above] {\scriptsize Accent Colors};
		\draw (0.3,1.6) -- (12.2,1.6);
		\draw [fill=Solarized-Yellow] (0,0) rectangle (1.3,1.3);
		\draw [fill=Solarized-Orange] (1.6,0) rectangle (2.9,1.3);
		\draw [fill=Solarized-Red] (3.2,0) rectangle (4.5,1.3);
		\draw [fill=Solarized-Magenta] (4.8,0) rectangle (6.1,1.3);
		\draw [fill=Solarized-Violet] (6.4,0) rectangle (7.7,1.3);
		\draw [fill=Solarized-Blue] (8,0) rectangle (9.3,1.3);
		\draw [fill=Solarized-Cyan] (9.6,0) rectangle (10.9,1.3);
		\draw [fill=Solarized-Green] (11.2,0) rectangle (12.5,1.3);
		\draw (0.65,0) node [anchor=north] {\scriptsize Yellow};
		\draw (2.25,0) node [anchor=north] {\scriptsize Orange};
		\draw (3.85,0) node [anchor=north] {\scriptsize Red};
		\draw (5.45,0) node [anchor=north] {\scriptsize Magenta};
		\draw (7.05,0) node [anchor=north] {\scriptsize Violet};
		\draw (8.65,0) node [anchor=north] {\scriptsize Blue};
		\draw (10.25,0) node [anchor=north] {\scriptsize Cyan};
		\draw (11.85,0) node [anchor=north] {\scriptsize Green};
	\end{tikzpicture}
	\caption{Farbschema solarized}
\end{figure}

Die Farben können verwendet werden, indem den Farbnamen unter den obigen Farbfeldern die Zeichenkette \texttt{Solarized-} vorangestellt wird. \\

\paragraph*{Beispiel:} \verb_{\color{Solarized-Blue}blauen}_ erzeugt einen {\color{Solarized-Blue}blauen} Text.



%---+----1----+----2----+----3----+----4----+----5----+----6----+----7----+----8
% vim:wrap:noet:ts=2 sw=2 sts=2

				\subsection{Apple (Finder) Tag-Farben}
					%
% File   : findertags.tex
% Author : ɛntiˈtɛːt.kaɪ̯
% Date   : 2016-03-20
%
%---+----1----+----2----+----3----+----4----+----5----+----6----+----7----+----8% LaTeX Metadaten
	\label{sec:FinderTags}
	\index{Tags}
	\index{Farbschema!Tags}
%---+----1----+----2----+----3----+----4----+----5----+----6----+----7----+----8



\begin{figure}[h]
	\centering
	\begin{tikzpicture}
		%\draw (5.45,1.6) -- (5.45,1.9) node [above] {\scriptsize Standard-Farben};
		%\draw (0.3,1.6) -- (10.6,1.6);
		\draw [fill=Finder-Tag-Red] (0,0) rectangle (1.3,1.3);
		\draw [fill=Finder-Tag-Orange] (1.6,0) rectangle (2.9,1.3);
		\draw [fill=Finder-Tag-Yellow] (3.2,0) rectangle (4.5,1.3);
		\draw [fill=Finder-Tag-Green] (4.8,0) rectangle (6.1,1.3);
		\draw [fill=Finder-Tag-Blue] (6.4,0) rectangle (7.7,1.3);
		\draw [fill=Finder-Tag-Violet] (8,0) rectangle (9.3,1.3);
		\draw [fill=Finder-Tag-Gray] (9.6,0) rectangle (10.9,1.3);
		\draw (0.65,0) node [anchor=north] {\scriptsize Red};
		\draw (2.25,0) node [anchor=north] {\scriptsize Orange};
		\draw (3.85,0) node [anchor=north] {\scriptsize Yellow};
		\draw (5.45,0) node [anchor=north] {\scriptsize Green};
		\draw (7.05,0) node [anchor=north] {\scriptsize Blue};
		\draw (8.65,0) node [anchor=north] {\scriptsize Violet};
		\draw (10.25,0) node [anchor=north] {\scriptsize Gray};
	\end{tikzpicture}
	\caption{Farben der Tags in Apples Finder}
\end{figure}
\FloatBarrier

Die Farben können verwendet werden, indem den Farbnamen unter den obigen Farbfeldern die Zeichenkette \texttt{Finder-Tag-} vorangestellt wird. \\

\paragraph*{Beispiel:} \verb_{\color{Finder-Tag-Blue}blauen}_ erzeugt einen {\color{Finder-Tag-Blue}blauen} Text.



%---+----1----+----2----+----3----+----4----+----5----+----6----+----7----+----8
% vim:wrap:noet:ts=2 sw=2 sts=2

			\section{Konfiguration des tabularx Pakets}
				\input{definitions/tabularx}
			\section{Konfiguration des listingsutf8 Pakets}
				\input{definitions/listingsutf8}

	% Schmutztitel (Rückseite)
		\includepdf{a4schmutztitel}



\end{document}



%---+----1----+----2----+----3----+----4----+----5----+----6----+----7----+----8
% vim:wrap:noet:ts=2 sw=2 sts=2
