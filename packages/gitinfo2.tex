%
% File   : gitinfo2.tex
% Author : ɛntiˈtɛːt.kaɪ̯
% Date   : 2016-03-20
%
%---+----1----+----2----+----3----+----4----+----5----+----6----+----7----+----8% LaTeX Metadaten
	\label{sec:Gitinfo2}
	\index{gitinfo2}
	\index{Paket!gitinfo2}
%---+----1----+----2----+----3----+----4----+----5----+----6----+----7----+----8



\begin{table}[b]
	\caption{Git Metadaten}
	\footnotesize
	\begin{tabularx}{\textwidth}{lFF} \hline
		Makro & Inhalt & Bemerkung \\ \hline
		\verb_\gitAbbrevHash_ &
			\gitAbbrevHash &
			Hash (7 Zeichen) \\
		\verb_\gitHash_ &
			\gitAbbrevHash\ldots &
			Hash (40 Zeichen, hier in verkürzter Darstellung) \\ \hline
		\verb_\gitAuthorName_ &
			\fontspec{Times New Roman}\gitAuthorName &
			Autor \\
		\verb_\gitAuthorEmail_ &
			\gitAuthorEmail &
			E-Mail (Autor) \\
		\verb_\gitAuthorDate_ &
			\gitAuthorDate &
			Datum (commit) \\
		\verb_\gitAuthorIsoDate_ &
			\gitAuthorIsoDate &
			ISO-Datum (commit) \\
		\verb_\gitAuthorUnixDate_ &
			\gitAuthorUnixDate &
			Unix-Datum (commit) \\ \hline
		\verb_\gitCommitterName_ &
			\fontspec{Times New Roman}\gitCommitterName &
			Committer \\
		\verb_\gitCommitterEmail_ &
			\gitCommitterEmail &
			E-Mail (Committer) \\
		\verb_\gitCommitterDate_ &
			\gitCommitterDate &
			Committer-Datum \\
		\verb_\gitCommitterIsoDate_ &
			\gitCommitterIsoDate &
			Committer-ISO-Datum \\
		\verb_\gitCommitterUnixDate_ &
			\gitCommitterUnixDate &
			Committer-Unix-Datum \\ \hline
		\verb_\gitReferences_ &
			\gitReferences &
			Referenzen (Tags, Branches) \\
		\verb_\gitBranch_ &
			\gitBranch &
			Aktueller Branch \\ \hline
		\verb_\gitVtag_ &
			\gitVtag &
			\\
		\verb_\gitVtags_ &
			\gitVtags &
			\\
		\verb_\gitVtagn_ &
			\gitVtagn &
			\\ \hline
		\verb_\gitFirstTagDescribe_ &
			\gitFirstTagDescribe &
			\\
		\verb_\gitDirty_ &
			\gitDirty &
			\\
		\verb_\gitRel_ &
			\gitRel &
			Release-Nummer \\
		\verb_\gitRels_ &
			\gitRels &
			\\
		\verb_\gitReln_ &
			\gitReln &
			\\
		\verb_\gitRoff_ &
			\gitRoff &
			Anzahl der commits zwischen HEAD und dem letzten Release \\
		\verb_\gitTags_ &
			\gitTags &
			Liste aller Tags \\
		\verb_\gitDescribe_ &
			\gitDescribe &
			\\ \hline
	\end{tabularx}
\end{table}
\FloatBarrier



%---+----1----+----2----+----3----+----4----+----5----+----6----+----7----+----8
% vim:wrap:noet:ts=2 sw=2 sts=2
