%
% File   : solarized.tex
% Author : ɛntiˈtɛːt.kaɪ̯
% Date   : 2016-03-20
%
%---+----1----+----2----+----3----+----4----+----5----+----6----+----7----+----8% LaTeX Metadaten
	\label{sec:Solarized}
	\index{solarized}
	\index{Farbschema!solarized}
%---+----1----+----2----+----3----+----4----+----5----+----6----+----7----+----8



Die folgenden 16~Farbfelder zeigen die Farben von Ethan Schoonovers \href{http://ethanschoonover.com/solarized}{Farbschema solarized}.

\begin{figure}[h]
	\centering
	\begin{tikzpicture}
		\draw (1.45,1.6) -- (1.45,1.9) node [above] {\scriptsize Background Tones};
		\draw (0.3,1.6) -- (2.6,1.6);
		\draw (6.25,1.6) -- (6.25,1.9) node [above] {\scriptsize Content Tones};
		\draw (3.5,1.6) -- (9,1.6);
		\draw (11.05,1.6) -- (11.05,1.9) node [above] {\scriptsize Background Tones};
		\draw (9.9,1.6) -- (12.2,1.6);
		\draw [fill=Solarized-Base03] (0,0) rectangle (1.3,1.3);
		\draw [fill=Solarized-Base02] (1.6,0) rectangle (2.9,1.3);
		\draw [fill=Solarized-Base01] (3.2,0) rectangle (4.5,1.3);
		\draw [fill=Solarized-Base00] (4.8,0) rectangle (6.1,1.3);
		\draw [fill=Solarized-Base0] (6.4,0) rectangle (7.7,1.3);
		\draw [fill=Solarized-Base1] (8,0) rectangle (9.3,1.3);
		\draw [fill=Solarized-Base2] (9.6,0) rectangle (10.9,1.3);
		\draw [fill=Solarized-Base3] (11.2,0) rectangle (12.5,1.3);
		\draw (0.65,0) node [anchor=north] {\scriptsize Base03};
		\draw (2.25,0) node [anchor=north] {\scriptsize Base02};
		\draw (3.85,0) node [anchor=north] {\scriptsize Base01};
		\draw (5.45,0) node [anchor=north] {\scriptsize Base00};
		\draw (7.05,0) node [anchor=north] {\scriptsize Base0};
		\draw (8.65,0) node [anchor=north] {\scriptsize Base1};
		\draw (10.25,0) node [anchor=north] {\scriptsize Base2};
		\draw (11.85,0) node [anchor=north] {\scriptsize Base3};
	\end{tikzpicture}

	~\\
	\begin{tikzpicture}
		\draw (6.25,1.6) -- (6.25,1.9) node [above] {\scriptsize Accent Colors};
		\draw (0.3,1.6) -- (12.2,1.6);
		\draw [fill=Solarized-Yellow] (0,0) rectangle (1.3,1.3);
		\draw [fill=Solarized-Orange] (1.6,0) rectangle (2.9,1.3);
		\draw [fill=Solarized-Red] (3.2,0) rectangle (4.5,1.3);
		\draw [fill=Solarized-Magenta] (4.8,0) rectangle (6.1,1.3);
		\draw [fill=Solarized-Violet] (6.4,0) rectangle (7.7,1.3);
		\draw [fill=Solarized-Blue] (8,0) rectangle (9.3,1.3);
		\draw [fill=Solarized-Cyan] (9.6,0) rectangle (10.9,1.3);
		\draw [fill=Solarized-Green] (11.2,0) rectangle (12.5,1.3);
		\draw (0.65,0) node [anchor=north] {\scriptsize Yellow};
		\draw (2.25,0) node [anchor=north] {\scriptsize Orange};
		\draw (3.85,0) node [anchor=north] {\scriptsize Red};
		\draw (5.45,0) node [anchor=north] {\scriptsize Magenta};
		\draw (7.05,0) node [anchor=north] {\scriptsize Violet};
		\draw (8.65,0) node [anchor=north] {\scriptsize Blue};
		\draw (10.25,0) node [anchor=north] {\scriptsize Cyan};
		\draw (11.85,0) node [anchor=north] {\scriptsize Green};
	\end{tikzpicture}
	\caption{Farbschema solarized}
\end{figure}

Die Farben können verwendet werden, indem den Farbnamen unter den obigen Farbfeldern die Zeichenkette \texttt{Solarized-} vorangestellt wird. \\

\paragraph*{Beispiel:} \verb_{\color{Solarized-Blue}blauen}_ erzeugt einen {\color{Solarized-Blue}blauen} Text.



%---+----1----+----2----+----3----+----4----+----5----+----6----+----7----+----8
% vim:wrap:noet:ts=2 sw=2 sts=2
